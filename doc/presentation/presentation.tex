\documentclass{beamer}
\usepackage[utf8]{inputenc}

\usepackage{listings}

%% https://github.com/nickgian/thesis/blob/master/lstcoq.sty
\usepackage{color}

\definecolor{ltblue}{rgb}{0,0.4,0.4}
\definecolor{dkblue}{rgb}{0,0.1,0.6}
\definecolor{dkgreen}{rgb}{0,0.35,0}
\definecolor{dkviolet}{rgb}{0.3,0,0.5}
\definecolor{dkred}{rgb}{0.5,0,0}

% lstlisting coq style (inspired from a file of Assia Mahboubi)
%
\lstdefinelanguage{Coq}{ 
%
% Anything betweeen $ becomes LaTeX math mode
mathescape=true,
%
% Comments may or not include Latex commands
texcl=false, 
%
% Vernacular commands
morekeywords=[1]{Section, Module, End, Require, Import, Export,
  Variable, Variables, Parameter, Parameters, Axiom, Hypothesis,
  Hypotheses, Notation, Local, Tactic, Reserved, Scope, Open, Close,
  Bind, Delimit, Definition, Let, Ltac, Fixpoint, CoFixpoint, Add,
  Morphism, Relation, Implicit, Arguments, Unset, Contextual,
  Strict, Prenex, Implicits, Inductive, CoInductive, Record,
  Structure, Canonical, Coercion, Context, Class, Global, Instance,
  Program, Infix, Theorem, Lemma, Corollary, Proposition, Fact,
  Remark, Example, Proof, Goal, Save, Qed, Defined, Hint, Resolve,
  Rewrite, View, Search, Show, Print, Printing, All, Eval, Check,
  Projections, inside, outside, Def},
%
% Gallina
morekeywords=[2]{forall, exists, exists2, fun, fix, cofix, struct,
  match, with, end, as, in, return, let, if, is, then, else, for, of,
  nosimpl, when},
%
% Sorts
morekeywords=[3]{Type, Prop, Set, true, false, option},
%
% Various tactics, some are std Coq subsumed by ssr, for the manual purpose
morekeywords=[4]{pose, set, move, case, elim, apply, clear, hnf,
  intro, intros, generalize, rename, pattern, after, destruct,
  induction, using, refine, inversion, injection, rewrite, setoid_rewrite, congr,
  unlock, compute, ring, field, fourier, replace, setoid_replace, fold, unfold,
  change, cutrewrite, simpl, have, suff, wlog, suffices, without,
  loss, nat_norm, assert, cut, trivial, revert, bool_congr, nat_congr,
  symmetry, transitivity, auto, split, left, right, autorewrite},
%
% Terminators
morekeywords=[5]{by, done, exact, reflexivity, tauto, romega, omega,
  assumption, solve, contradiction, discriminate},
%
% Control
morekeywords=[6]{do, last, first, try, idtac, repeat},
%
% Comments delimiters, we do turn this off for the manual
morecomment=[s]{(*}{*)},
%
% Spaces are not displayed as a special character
showstringspaces=false,
%
% String delimiters
morestring=[b]",
morestring=[d]’,
%
% Size of tabulations
tabsize=3,
%
% Enables ASCII chars 128 to 255
extendedchars=false,
%
% Case sensitivity
sensitive=true,
%
% Automatic breaking of long lines
breaklines=false,
%
% Default style fors listings
basicstyle=\small,
%
% Position of captions is bottom
captionpos=b,
%
% flexible columns
basewidth={2em, 0.5em},
columns=flexible,
%
% Style for (listings') identifiers
identifierstyle={\ttfamily\color{black}},
% Style for declaration keywords
keywordstyle=[1]{\ttfamily\bfseries\color{dkviolet}},
% Style for gallina keywords
keywordstyle=[2]{\ttfamily\bfseries\color{dkgreen}},
% Style for sorts keywords
keywordstyle=[3]{\ttfamily\bfseries\color{ltblue}},
% Style for tactics keywords
keywordstyle=[4]{\ttfamily\color{dkblue}},
% Style for terminators keywords
keywordstyle=[5]{\ttfamily\color{dkred}},
%Style for iterators
%keywordstyle=[6]{\ttfamily\color{dkpink}},
% Style for strings
stringstyle=\ttfamily,
% Style for comments
commentstyle={\ttfamily\itshape\color{dkgreen}},
%
%moredelim=**[is][\ttfamily\color{red}]{/&}{&/},
literate=
    {fun}{{\color{dkgreen}{$\lambda\;$}}}1
    {bool}{{$\mathbb{B}$}}1
    {nat}{{$\mathbb{N}$}}1
    {Vforall2}{Vforall2}1 % quick workardoun to avoid partial replacement of 'forall' in identifier
    {nat\_equiv}{nat\_equiv}1 % quick workardoun to avoid partial replacement of 'nat' in identifier
    {forall}{{\color{dkgreen}{$\forall\;$}}}1
    {exists}{{$\exists\;$}}1
    {<-}{{$\leftarrow\;\;$}}1
    {=>}{{$\Rightarrow\;\;$}}1
    {==}{{\texttt{==}\;}}1
    {==>}{{$\Longrightarrow\;\;$}}1
%    {:>}{{\texttt{:>}\;}}1
    {->}{{$\rightarrow\;\;$}}1
    {<-->}{{$\longleftrightarrow\;\;$}}1
    {<->}{{$\leftrightarrow\;\;$}}1
    {<==}{{$\leq\;\;$}}1
    {\#}{{$^\star$}}1 
    {\\o}{{$\circ\;$}}1 
%    {\@}{{$\cdot$}}1 
    {\/\\}{{$\wedge\;$}}1
    {\\\/}{{$\vee\;$}}1
    {++}{{\texttt{++}}}1
    {~}{{\ }}1
    {¬}{{$\lnot$}}1     % this does not work
    {\@\@}{{$@$}}1
    {\\mapsto}{{$\mapsto\;$}}1
    {\\hline}{{\rule{\linewidth}{0.5pt}}}1
%
}[keywords,comments,strings]

\lstnewenvironment{coq}{\lstset{language=Coq}}{}

% pour inliner dans le texte
\def\coqe{\lstinline[language=Coq, basicstyle=\small]}
% pour inliner dans les tableaux / displaymath...
\def\coqes{\lstinline[language=Coq, basicstyle=\scriptsize]}

%%% Local Variables: 
%%% mode: latex
%%% Local IspellDict: british
%%% TeX-master: "main.tex"
%%% End: 

\title{Formal Verification of C programs using C light Operational Semantics}
\author{Nika Pona \\
  Digamma.ai}



\begin{document}

\maketitle

\begin{frame}{Outline}
\begin{enumerate}
\item Formal verification - quick intro (high-level)
\item Coq mini intro 
\item CompCert

\item Reasoning about C programs in Coq
  \begin{itemize}
    \item C light syntax
   \item Operational Semantics
\end{itemize}
\item Toy example: strlen 
\begin{itemize}
\item Informal specification 
\item Formal specification of strlen 
\item Simple implementation in C
\item From C program to AST using clightgen
\item Correctness proof
%\item Undefined behaviours in C and guarding against them
\end{itemize}
\item Conclusions
\end{enumerate} 
\end{frame}

% do the first two sections in the end, maybe ask boys
\section{Formal verification - quick intro}
\begin{frame}{Formal verification - quick intro}
 
  We want to have high assurance that our code works as intended. One of the methods is formal verification. This means we want to produce a formal proof that our code works as intended. What does it mean exactly and how do we do it?
  
  \begin{enumerate}
  \item We start by writing a \emph{specification} in a formal language (like Coq or HOL) which strictly defines how our program should behave.
  \item Then we derive the \emph{semantics} of our program which describes how it actually behaves. 
  \item Finally we mathematically prove that semantics of our program matches our specification.
\end{enumerate}
  
\end{frame}

\begin{frame}{Coq intro}
  As a formal language we choose {\bf Gallina}, mechanized version of Calculus of Inductive Constructions (aka dependent type theory), which is a very expressive theory well studied in mathematical logic.


  \smallskip
  
  It is much more likely to make a mistake in a formal proof (which is typically way longer than the code), so we want assurance that our proof is correct.
  \smallskip
  
  Hence we use a proof assistant Coq: a program that checks that your proof is correct. It also provides an environment to make the construction of proofs easier. Coq's language is based on dependent type theory and is called Gallina.
  
\end{frame}
\begin{frame}{Coq intro}
  
 [Show some basic Coq definitions and proofs]
  
  \end{frame}

  \begin{frame}{CompCert}
    Coq has been used to conduct some big verification projects. One of them is CompCert, a verfied compiler for C, almost entirely written in Coq and proved to work according to the specification (\url{http://compcert.inria.fr/}).

    \begin{quote}
      The striking thing about our CompCert results is that the middle-end bugs we found in all other compilers are absent. As of early 2011,the under-development version of CompCert is the only compiler we have tested for which Csmith cannot find wrong-code errors.
    \end{quote} ({\it Finding and Understanding Bugs in C Compilers}, Yang et al., 2011)

  \end{frame}
  \begin{frame}{CompCert}
    
 To achieve this they formalized C syntax and semantics (C99 standard).

    \bigskip
    
    Nice thing about Coq is that writing a specification is basically the same as writing a program that meets that specification, since Gallina is a functional programming language. One can extract the code to OCaml or Haskell to compile and run it.

    
    
\end{frame}


\section{Approach}

\begin{frame}{Traditional approach}
   A function that needs to be verified can be written and proven in Coq directly. Then the Coq code can be extracted to OCaml.
   
   This approach, while being (possibly) easier, has some downsides:
\begin{itemize}
  \item Extracted OCaml code shows low performance
  \item Coq's extraction mechanism is itself not verified so it can produce bugs uless restricted to ML subset of gallina
  \item The extracted portion generally requires a pure-OCaml wrapper to be used with real-life data. Therefore, extraction necessarily adds a layer that cannot be verified - only unit-tested
\end{itemize}
\end{frame}
   
\begin{frame}{New approach}
  We decided to try to verify the implementation of ASN1C compiler that already exists. This reduces TCB and moreover we could use the same techniques in other projects. We reuse parts of CompCert for this.
\begin{itemize}
  \item parse C code into an abstract syntax tree using C light generator of CompCert (not verified)
  \item write a functional specification using CompCert's model of C light 
  \item reason about the C light program using operational semantics defined in CompCert
\end{itemize}
\end{frame}

%\begin{frame}
 % Concrete vs Abstract syntax

  %We write a C program in concrete C syntax, which is designed to be used by a parser (a + b).
  %Abstract syntax tree: nodes are constructors, leaves are atoms (plus (a,b)).
  %todo: more on AST

  %Deep embedding of C light to Coq := the abstract syntax is defined as inductive datatypes
  
%\end{frame}

\begin{frame}{Back to CompCert}
  CompCert formalizes C syntax and its semantics resulting in CompCert C and C light languages. In particular, they formalize C integers and memory model. I will briefly explain these since they are used throught whole development.
 \end{frame}
\section{Libraries of CompCert}

\begin{frame}

  \begin{center}
    \huge Libraries of CompCert
    \end{center}
  \end{frame}

\begin{frame}[t,fragile]{CompCert Integers}

  Machine integers modulo $2^N$ are defined as a module type in \url{CompCert/lib/Integers.v}. $8, 32, 64$-bit integers are supported, as well as 32 and 64-bit pointer offsets.\\
  
\bigskip

A machine integer (type \texttt{int}) is represented as a Coq arbitrary-precision
integer (type \texttt{Z} ) plus a proof that it is in the range 0 (included) to
modulus (excluded).


\begin{lstlisting}[language=Coq]
Record int: Type :=
mkint { intval: Z; intrange: -1 < intval < modulus }.
\end{lstlisting}

The function \texttt{repr} takes a Coq integer and returns the corresponding machine integer. The argument is treated modulo \texttt{modulus}.

\end{frame}

\begin{frame}[t,fragile]{CompCert Integers}

 Integer is basically a natural number with a bound, thus we can prove an induction principle for integers

\bigskip

\begin{lstlisting}[language=Coq]
  Lemma int_induction :
     forall (P : int -> Prop), P Int.zero ->
          (forall i, P i -> P (Int.add i Int.one)) ->
                 forall i, P i.
\end{lstlisting}

 \begin{proof}

   By using induction principle for non-negative integers \texttt{natlike\_ind} for \texttt{Z}.
                                       
 \end{proof}

 They prove some basic arithmetic theorems available through hints \texttt{ints} and \texttt{ptrofs}.

\end{frame}

\begin{frame}[fragile]{Memory Model}

  A memory model is a specification of memory states and operations over memory. Cf. \url{CompCert/common/Memtype.v} (interface of the memory model)
  
\bigskip
  
   Memory states are accessed by addresses, pairs of a block
  identifier $b$ and a byte offset $ofs$ within that block.
  Each address is associated to permissions (current and maximal): \texttt{Freeable, Writable, Readable, Nonempty, Empty}, ranging from allowing all operations to allowing no operation respectively.
\end{frame}

\begin{frame}[fragile]{Memory Type}

Implementation of the memory type in \url{CompCert/common/Memory.v}: 
  
  \begin{lstlisting}[language=Coq]
Record mem : Type := mkmem {
  mem_contents: PMap.t (ZMap.t memval);  
  mem_access: PMap.t (Z -> perm_kind -> option permission);
  nextblock: block;
  access_max: forall b ofs, perm_order'' (mem_access#b ofs Max)
                            (mem_access#b ofs Cur);
  nextblock_noaccess: forall b ofs k, ~(Plt b nextblock) ->
                            mem_access#b ofs k = None;
  contents_default: forall b, fst mem_contents#b = Undef }.

\end{lstlisting}


\end{frame}

\begin{frame}{Memory Model}
  The type \texttt{mem} of memory states has  the following 4 basic operations over memory states:
  \bigskip
  
\begin{itemize}
\item [load]: read a memory chunk at a given address;
\item [store]: store a memory chunk at a given address;
\item [alloc]: allocate a fresh memory block;
\item [free]: invalidate a memory block.
\end{itemize}
\bigskip
A load succeeds if and only if the access is valid for reading. The value returned by \texttt{load} belongs to the type of the memory quantity accessed etc. 



\end{frame}

\begin{frame}{Applicative finite maps}
  Memory model and environments used in evaluation of the statements are modelled as applicative finite maps. Cf. \url{CompCert/lib/Maps.v}

  \bigskip
  The two main operations are \texttt{set k d m}, which returns a map identical to \texttt{m} except that \texttt{d}
  is associated to \texttt{k}, and \texttt{get k m} which returns the data associated
  to key \texttt{k} in map \texttt{m}.  There are two distinguish two kinds of maps:
  \begin{enumerate}
\item  the \texttt{get} operation returns an option type, either \texttt{None} 
  \item 
the {get} operation always returns a data.  If no data was explicitly
  associated with the key, a default data provided at map initialization time
  is returned.
  \end{enumerate}
  \end{frame}


\section{C light syntax}
\begin{frame}
  \begin{center}
    \huge C light Syntax
    \end{center}
  \end{frame}


\begin{frame}{C light types}
  Each expression of CompCert C has a type. There are:
  \begin{itemize}
   \item numeric types (integers and floats)
   \item pointers
   \item arrays
   \item function types
     \item composite types (struct and union).
 \end{itemize}
  An integer type is a pair of a signed/unsigned
  flag and a bit size: 8, 16, or 32 bits, or the special IBool size
  standing for the C99 Bool type.  64-bit integers are treated separately.
  

\end{frame}

\begin{frame}[fragile]{C light types}

  Cf. \url{CompCert/cfrontend/Ctypes.v}. 
  \begin{lstlisting}[language=Coq]
 Inductive type : Type :=
  | Tvoid: type 
  | Tint: intsize -> signedness -> attr -> type 
  | Tlong: signedness -> attr -> type 
  | Tfloat: floatsize -> attr -> type 
  | Tpointer: type -> attr -> type       
  | Tarray: type -> Z -> attr -> type              
  | Tfnction: typelist -> type -> calling_convention -> type   
  | Tstruct: ident -> attr -> type                 
  | Tunion: ident -> attr -> type                 
\end{lstlisting}


\end{frame}

\begin{frame}[fragile]{C light types}
  C light types are subset of CompCert C types, we ignore the attributes.
  \begin{lstlisting}[language=Coq]
Definition tvoid := Tvoid.
Definition tschar := Tint I8 Signed noattr.
Definition tuchar := Tint I8 Unsigned noattr.
Definition tshort := Tint I16 Signed noattr.
Definition tushort := Tint I16 Unsigned noattr.
Definition tint := Tint I32 Signed noattr.
Definition tuint := Tint I32 Unsigned noattr.
Definition t$\texttt{bool}$ := Tint IBool Unsigned noattr.
Definition tlong := Tlong Signed noattr.
Definition tulong := Tlong Unsigned noattr.
Definition tfloat := Tfloat F32 noattr.
Definition tdouble := Tfloat F64 noattr.
Definition tptr (t: type) := Tpointer t noattr.
Definition tarray (t: type) (sz: Z) := Tarray t sz noattr.
\end{lstlisting}

\end{frame}

\begin{frame}{C light Expressions}
  \begin{itemize}
  \item (long) integer constant
  \item double/single float constant
  \item (temporary) variable
  \item pointer dereference ($*$)
  \item address-of operator ($\&$)
  \item unary operation
  \item binary operation
  \item type cast
  \item access to a member of a struct or union
  \item size of a type
    \item alignment of a type
    \end{itemize}
  \end{frame}

  \begin{frame}{C light Expressions}
    All expressions and their sub-expressions
are annotated by their static types. Within expressions, only side-effect free operators are supported, moreover, assignments and
function calls are statements and cannot occur within expressions.
\bigskip

As a
consequence, all Clight expressions always terminate and are pure: their evaluation
have no side effects. This ensures determinism of evaluation.
    \end{frame}

 \begin{frame}[fragile]{C light Expressions}
   
  \begin{lstlisting}[language=Coq]
Inductive expr : Type :=
| Econst_int: int -> type -> expr      
| Econst_float: float -> type -> expr 
| Econst_single: float32 -> type -> expr 
| Econst_long: int64 -> type -> expr 
| Evar: ident -> type -> expr 
| Etempvar: ident -> type -> expr 
| Ederef: expr -> type -> expr 
| Eaddrof: expr -> type -> expr
| Eunop: unary_operation -> expr -> type -> expr
| Ebinop: binary_operation -> expr -> expr -> type -> expr
| Ecast: expr -> type -> expr   
| Efield: expr -> ident -> type -> expr
| Esizeof: type -> type -> expr 
| Ealignof: type -> type -> expr
}. 
\end{lstlisting}\footnote{\url{CompCert/cfrontend/Clight.v}}


\end{frame}


    

\begin{frame}[fragile]{C light Expressions: Examples}

  \begin{lstlisting}[language=Coq]
    (* 0 *)
    (Econst_int Int.zero tint) 

    (* 0 + 1 *)
    (Ebinop Oadd (Econst_int Int.zero tint)
    (Econst_int (Int.repr 1) tint) (tint))

    (* int *p *)
    (Etempvar _p (tptr tint)) 
    
    (* (*p) *)
    (Ederef (Etempvar _p (tptr tint)) tint)


  \end{lstlisting}
  


\end{frame}

\begin{frame}[fragile]{C light Statements}

  \begin{lstlisting}[language=Coq]
Inductive statement : Type :=
| Sskip : statement (* do nothing *)
| Sassign : expr -> expr -> statement
(* assignment lvalue = rvalue *)
| Sset : ident -> expr -> statement
(* assignment tempvar = rvalue *)
| Scall: option ident -> expr -> list expr -> statement
| Sbuiltin: option ident -> external_$\texttt{fun}$ction -> typelist
-> list expr -> statement (* builtin invocation *)
| Ssequence : statement -> statement -> statement
| Sifthenelse : expr  -> statement -> statement -> statement
| Sloop: statement -> statement -> statement (* infinite loop *)
| Sbreak : statement
| Scontinue : statement
| Sreturn : option expr -> statement
| Sswitch : expr -> labeled_statements -> statement
| Slabel : label -> statement -> statement
| Sgoto : label -> statement

  \end{lstlisting}
  


\end{frame}



\begin{frame}[fragile]{C light Statements}

  C loops are derived:

  \begin{lstlisting}[language=Coq]
    
Definition Swhile (e: expr) (s: statement) :=
  Sloop (Ssequence (Sifthenelse e Sskip Sbreak) s) Sskip.

Definition Sdowhile (s: statement) (e: expr) :=
  Sloop s (Sifthenelse e Sskip Sbreak).

  Definition Sfor (s1: statement) (e2: expr) (s3: statement)
  (s4: statement) := Ssequence s1
   (Sloop (Ssequence (Sifthenelse e2 Sskip Sbreak) s3) s4).

  \end{lstlisting}
  


\end{frame}


\begin{frame}[fragile]{C light Statements: Examples}

  \begin{lstlisting}[language=Coq]
     (* int s = 1; *)
      (Sset _s (Econst_int (Int.repr 1) tint))

      (* return s; *)
      (Sreturn (Some (Etempvar _s tint)))

      (* while (s) {s = s - 1;} *)
      (Swhile (Etempvar _s tint) 
      (Ssequence  
        (Sset _s (Ebinop Osub (Etempvar _input tint)
                (Econst_int (Int.repr 1) tint) tint))))


  \end{lstlisting}
  


\end{frame}

\begin{frame}{Unsupported Features}
  \begin{itemize}
  \item \texttt{extern} declaration of arrays
\item structs and unions cannot be passed by value
\item  type qualifiers (\texttt{const}, \texttt{volatile}, \texttt{restrict}) are erased at parsing
\item within expressions no side-effects nor function calls (meaning all C light expressions always terminate and are pure)
\item statements: in \texttt{for(s1, a, s2)} s1 and s2 are statements, that do not terminate by break
\item \texttt{extern} functions are only declared and not defined, used to model system calls
\end{itemize}
  
  \end{frame}


  \section{Operational semantics: bigstep}
  \begin{frame}
  \begin{center}
    \huge Semantics of C light
    \end{center}
  \end{frame}

  \begin{frame}{Operational Semantics}
    Our goal is to prove that programs written in C light behave as intented. To do this we need to formalize the notion of meaning of a C light program. We do this using what is called {\bf operational semantics}.

    \bigskip
    
    An operational semantics is a mathematical model of programming language execution. It is basically an interpreter defined formally.
    \bigskip
    
    Here I will talk about the formalization of big-step operational semantics used for all intermediate languages of CompCert.
  \end{frame}
  \begin{frame}{Operational Semantics}
 
    Each syntactic element (expressions and statements) is related to the indended result of executing this element.

    \bigskip
    
    Expressions are deterministically mapped to memory locations or values (integers, bool etc).

    \bigskip 
    The execution of statements depends on memory state and values stored in the local environment and produces {\bf outcomes} (break, normal, return), updated memory and local environment. Moreover, {\bf trace} of external calls is recorded\footnote{       (cf. \url{CompCert/cfrontend/ClightBigstep.v}).}.
     
    \end{frame}

    \begin{frame}{Semantic Elements: Values}
      
      We assign primitive values to constants and then compositionally compute values of expressions and outcomes of statements.

      \bigskip
      
      A CompCert C value is either:
      
    \begin{itemize}
    
\item a machine integer;
\item a floating-point number;
\item a pointer: a pair of a memory address and an integer offset with respect
  to this address;
\item the \texttt{Vundef} value denoting an arbitrary bit pattern, such as the
  value of an uninitialized variable.
 \end{itemize}
\end{frame}

\begin{frame}[t,fragile]{Semantic Elements: Values}

 Definition of values in Coq: \url{CompCert/common/Values.v}
  
 \begin{lstlisting}[language=Coq]
   
  Inductive val: Type :=
  | Vundef: val
  | Vint: int -> val
  | Vlong: int64 -> val
  | Vfloat: float -> val
  | Vsingle: float32 -> val
  | Vptr: block -> ptrofs -> val.
\end{lstlisting}



\begin{itemize}
\item \texttt{float} type is formalized in Flocq library
\item \texttt{int} and \texttt{ptrofs} types are defined in CompCert's Integers library
\end{itemize}
  
\end{frame}  











  

\begin{frame}[fragile]{Evaluation of expressions}
 
The evaluation of constants is straightforward: map to the same integer/float value:
  
    \begin{lstlisting}[language=Coq]
   
  Inductive eval_expr: expr -> val -> Prop :=
  | eval_Econst_int:   forall i ty,
      eval_expr (Econst_int i ty) (Vint i)
  | eval_Econst_float:   forall f ty,
      eval_expr (Econst_float f ty) (Vfloat f)
  | eval_Econst_single:   forall f ty,
      eval_expr (Econst_single f ty) (Vsingle f)
  | eval_Econst_long:   forall i ty,
      eval_expr (Econst_long i ty) (Vlong i)
\end{lstlisting}
\end{frame}


\begin{frame}[fragile]{Evaluation of expressions}
 
The evaluation of constants is straightforward: map to the same integer/float value:
  
    \begin{lstlisting}[language=Coq]
   
 | eval_Etempvar:  forall id ty v,
      le!id = Some v ->
      eval_expr (Etempvar id ty) v
\end{lstlisting}
\end{frame}

 \begin{frame}[fragile]{Evaluation of expressions}

  For unary and binary expression, the semantics of each operation is defined and applied to the values of the operands:
  
    \begin{lstlisting}[language=Coq]
   
 | eval_Eunop:  forall op a ty v1 v,
      eval_expr a v1 ->
      sem_unary_operation op v1 (typeof a) m = Some v ->
      eval_expr (Eunop op a ty) v
\end{lstlisting}
\end{frame}


 \begin{frame}[fragile]{Evaluation of expressions}

\texttt{eval\_lvalue ge e m a b ofs} defines the evaluation of expression a
  in l-value position given global and local environments and memory m.  The result is the memory location $[b, ofs]$
  that contains the value of the expression a.
  
    \begin{lstlisting}[language=Coq]
   
 eval_lvalue: expr -> block -> ptrofs -> Prop :=
  | eval_Evar_local:   forall id l ty,
      e!id = Some(l, ty) ->
      eval_lvalue (Evar id ty) l Ptrofs.zero
  | eval_Evar_global: forall id l ty,
      e!id = None ->
      Genv.find_symbol ge id = Some l ->
      eval_lvalue (Evar id ty) l Ptrofs.zero
  | eval_Ederef: forall a ty l ofs,
      eval_expr a (Vptr l ofs) ->
      eval_lvalue (Ederef a ty) l ofs
\end{lstlisting}
\end{frame}


\begin{frame}{Execution of statements}
  
     \texttt{exec\_stmt ge e m1 s t m2 out} describes the execution of
  the statement \texttt{s}.  \texttt{out} is the outcome for this execution.
  \texttt{m1} is the initial memory state, \texttt{m2} the final memory state.
  \texttt{t} is the trace of input/output events performed during this
  evaluation\footnote{\url{CompCert/cfrontend/ClightBigstep.v}}
\end{frame}

\begin{frame}[fragile]{Evaluation of expressions}

  ``Do nothing'' always evaluated to normal outcome: 

  \begin{lstlisting}[language=Coq]
   
  | exec_Sskip:   forall e le m,
      exec_stmt e le m Sskip
               E0 le m Out_normal
\end{lstlisting}

Setting a value $v$ to some $id$ results in modified temporary environment:
  
    \begin{lstlisting}[language=Coq]
   
 | exec_Sset:     forall e le m id a v,
      eval_expr ge e le m a v ->
      exec_stmt e le m (Sset id a)
               E0 (PTree.set id v le) m Out_normal
\end{lstlisting}
\end{frame}

\begin{frame}[fragile]{Evaluation of expressions}

  A more complicated example: loop. If we reached a break or return on the first (or second) statement, then the loop output is normal or return:
  
  \begin{lstlisting}[language=Coq]
   
   | exec_Sloop_stop1: forall e le m s1 s2 t le' m' out' out,
      exec_stmt e le m s1 t le' m' out' ->
      out_break_or_return out' out ->
      exec_stmt e le m (Sloop s1 s2) t le' m' out
\end{lstlisting}

Or we loop again if both statements result in normal outcome:
  
    \begin{lstlisting}[language=Coq]
   
      | exec_Sloop_loop: forall e le m s1 s2 t1 le1 m1 out1 t2 le2
      m2 t3 le3 m3 out,
      exec_stmt e le m s1 t1 le1 m1 out1 ->
      out_normal_or_continue out1 ->
      exec_stmt e le1 m1 s2 t2 le2 m2 Out_normal ->
      exec_stmt e le2 m2 (Sloop s1 s2) t3 le3 m3 out ->
      exec_stmt e le m (Sloop s1 s2)
                (t1**t2**t3) le3 m3 out
\end{lstlisting}
\end{frame}

\begin{frame} To see operational semantics in action, see factorial example in \url{asn1fpcoq/c/poc/factorial}.
  \end{frame}

           

  \begin{frame}
  \begin{center}
    \huge Toy example: \texttt{strlen} function
  \end{center}
  \end{frame}

  \begin{frame}{Toy example: \texttt{strlen} function}
    A more interesting example that involves memory model is \texttt{strlen} function that calculates the length of a C string.
    \end{frame}

\begin{frame}{Informal Specification}


  From the GNU C Reference Manual:
 \bigskip


  $\ldots$ A string constant is of type ``array of characters''. All string constants contain a null termination character as their last character.

\bigskip

$\ldots$ The strlen() function calculates the length of the string pointed to
 by s, excluding the terminating null byte.

 \bigskip

 
$\ldots$ 
 The strlen() function returns the number of bytes in the string
 pointed to by s.

 \bigskip

Conforming to $\ldots$ C99, C11, SVr4, 4.3BSD.



\end{frame}

\begin{frame}{Simple C Implementation}

\lstinputlisting[language=C]{strlen.c}

\end{frame}

\begin{frame}[t,fragile]{Formal Specification}

Here we use relational specification. It has an advantage that we only describe what is true and we don't have to worry about termination. Moreover, we can use \texttt{inversion} to prove lemmas about the spec. Alternatively, we could specify the same function using a functional spec. 
  \begin{lstlisting}[language=Coq]
    
Inductive strlen (m : mem) (b : block) (ofs : ptrofs) : int -> Prop :=
  | LengthZero: load m [b,ofs] = Some 0 -> strlen m b ofs 0
  | LengthSucc: forall (n : int) (c : char),
                      strlen m b ofs + 1 n ->
                      load m [b,ofs] = Vint c ->
                      c <> Int.zero ->
                      n + 1 <= UINT MAX ->
                      strlen m b ofs n + 1.
                    \end{lstlisting}
                    
Note that we are guarding against integer overflow here.  
  
\end{frame}

\begin{frame}[fragile]{Formal Specification}

 
  \begin{lstlisting}[language=Coq]
    
Fixpoint strlen_$fun$ (m : mem) (b : block) (ofs : Z) (l: int)
    (intrange : nat) {struct intrange} : option int:= 
      match intrange with
      | O => None (* out of int range *)
      | S n => match load m b ofs with 
              | Some v =>
                if is_null v
                then Some l else strlen_$fun$ m b (ofs + 1) (l + 1) n  
              | None => None 
              end
      end.

      Definition strlen_$fun$_spec (m : mem) (b: block) (ofs : Z)
      :=  strlen m b ofs 0 INTSIZE.
                    \end{lstlisting}
                    
                    %Then we could show that the functional spec is equivalent to the relational spec.

                    % \begin{lstlisting}[language=Coq]    
%Lemma strlen_$fun$_corr:
     % strlen_$fun$_spec m b ofs = Some n <-> strlen m b ofs n. 
      
               %     \end{lstlisting}
                    
\end{frame}


\begin{frame}[fragile]{C light AST (loop of strlen)}

\begin{lstlisting}[language=Coq]
Definition f_strlen_loop := 
(Sloop (Ssequence
(Ssequence
 (Ssequence
  (Sset _t1 (Etempvar _s (tptr tuchar)))
  (Sset _s
    (Ebinop Oadd (Etempvar _t1 (tptr tuchar))
      (Econst_int (Int.repr 1) tint) (tptr tuchar))))
 (Ssequence
  (Sset _t2 (Ederef (Etempvar _t1 (tptr tuchar)) tuchar))
  (Sifthenelse (Etempvar _t2 tuchar) Sskip Sbreak)))
(Sset _i
(Ebinop Oadd (Etempvar _i tuint) (Econst_int (Int.repr 1)
 tint)
  tuint))) Sskip) |}.

\end{lstlisting}

\end{frame}

\begin{frame}[fragile]{Correctness}

  We prove that for all strings our program computes correct result. In particular:
  
  \begin{theorem}
    For all addresses $[b, ofs]$ where a valid C string of length $len$ is stored, the C light AST $f\_strlen$ evaluates to $len$.
  \end{theorem}
  
  
  \begin{lstlisting}[language=Coq]
Lemma strlen_correct:
forall len m b ofs le, strlen m b ofs len -> exists t l',
  le!_input = Some (Vptr b ofs) ->
  exec_stmt le m f_strlen t le' m (Out_return (Some (Vint len))).

\end{lstlisting}

To prove this statement we have to prove that loop works correctly.

 



\end{frame}

  
\begin{frame}[fragile]{Correctness cont'd}

  
  
  \begin{lstlisting}[language=Coq]
 Lemma strlen_loop_correct: forall len m b ofs le,
 strlen m b ofs len -> exists t le',
 le!_output = Some (Vint 0) ->
 le!_input = Some (Vptr b ofs) ->
 exec_stmt ge e le m f_strlen_loop t le' m Out_normal
                     /\ le'!_output = Some (Vint len).
\end{lstlisting}

\begin{proof}
  We prove a generalization of this statement
  
 \begin{lstlisting}[language=Coq]
 Lemma strlen_loop_correct_gen : forall len m b ofs le,
 strlen m b ofs + i len -> exists t le',
 le!_output = Some (Vint i) ->
 le!_input = Some (Vptr b ofs + i) ->
 exec_stmt ge e le m f_strlen_loop t le' m Out_normal
                     /\ le'!_output = Some (Vint len + i).
\end{lstlisting}

by \texttt{int}-induction on $len$ and $i$. 

  \end{proof}

 



\end{frame}

\begin{frame}{Conclusion}
  Thus we have proved that on all strings of length smaller than
  \texttt{UINT\_MAX}, \texttt{strlen} works correctly. 
  
  \end{frame}

 

\begin{frame}
    \begin{thebibliography}{9}

 \bibitem{blazy} 
    {Sandrine Blazy and Xavier Leroy},
    {Mechanized Semantics for the Clight Subset of the C Language},
    {2009}
      
\bibitem{leroy}
    Xavier Leroy and Sandrine Blazy and Gordon Stewart,
    {The CompCert Memory Model, Version 2},
    {2012}

   

\end{thebibliography}
  
  \end{frame}
    


\end{document}
