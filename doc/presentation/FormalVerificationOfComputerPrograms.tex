\documentclass[10pt]{beamer}

\usetheme[progressbar=frametitle]{metropolis}
\usepackage{appendixnumberbeamer}

\usepackage{booktabs}
\usepackage[scale=2]{ccicons}

\usepackage{pgfplots}
\usepgfplotslibrary{dateplot}

\usepackage{xspace}
\newcommand{\themename}{\textbf{\textsc{metropolis}}\xspace}

\usepackage{listings}

%% https://github.com/nickgian/thesis/blob/master/lstcoq.sty
\usepackage{color}

\definecolor{ltblue}{rgb}{0,0.4,0.4}
\definecolor{dkblue}{rgb}{0,0.1,0.6}
\definecolor{dkgreen}{rgb}{0,0.35,0}
\definecolor{dkviolet}{rgb}{0.3,0,0.5}
\definecolor{dkred}{rgb}{0.5,0,0}

% lstlisting coq style (inspired from a file of Assia Mahboubi)
%
\lstdefinelanguage{Coq}{ 
%
% Anything betweeen $ becomes LaTeX math mode
mathescape=true,
%
% Comments may or not include Latex commands
texcl=false, 
%
% Vernacular commands
morekeywords=[1]{Section, Module, End, Require, Import, Export,
  Variable, Variables, Parameter, Parameters, Axiom, Hypothesis,
  Hypotheses, Notation, Local, Tactic, Reserved, Scope, Open, Close,
  Bind, Delimit, Definition, Let, Ltac, Fixpoint, CoFixpoint, Add,
  Morphism, Relation, Implicit, Arguments, Unset, Contextual,
  Strict, Prenex, Implicits, Inductive, CoInductive, Record,
  Structure, Canonical, Coercion, Context, Class, Global, Instance,
  Program, Infix, Theorem, Lemma, Corollary, Proposition, Fact,
  Remark, Example, Proof, Goal, Save, Qed, Defined, Hint, Resolve,
  Rewrite, View, Search, Show, Print, Printing, All, Eval, Check,
  Projections, inside, outside, Def},
%
% Gallina
morekeywords=[2]{forall, exists, exists2, fun, fix, cofix, struct,
  match, with, end, as, in, return, let, if, is, then, else, for, of,
  nosimpl, when},
%
% Sorts
morekeywords=[3]{Type, Prop, Set, true, false, option},
%
% Various tactics, some are std Coq subsumed by ssr, for the manual purpose
morekeywords=[4]{pose, set, move, case, elim, apply, clear, hnf,
  intro, intros, generalize, rename, pattern, after, destruct,
  induction, using, refine, inversion, injection, rewrite, setoid_rewrite, congr,
  unlock, compute, ring, field, fourier, replace, setoid_replace, fold, unfold,
  change, cutrewrite, simpl, have, suff, wlog, suffices, without,
  loss, nat_norm, assert, cut, trivial, revert, bool_congr, nat_congr,
  symmetry, transitivity, auto, split, left, right, autorewrite},
%
% Terminators
morekeywords=[5]{by, done, exact, reflexivity, tauto, romega, omega,
  assumption, solve, contradiction, discriminate},
%
% Control
morekeywords=[6]{do, last, first, try, idtac, repeat},
%
% Comments delimiters, we do turn this off for the manual
morecomment=[s]{(*}{*)},
%
% Spaces are not displayed as a special character
showstringspaces=false,
%
% String delimiters
morestring=[b]",
morestring=[d]’,
%
% Size of tabulations
tabsize=3,
%
% Enables ASCII chars 128 to 255
extendedchars=false,
%
% Case sensitivity
sensitive=true,
%
% Automatic breaking of long lines
breaklines=false,
%
% Default style fors listings
basicstyle=\small,
%
% Position of captions is bottom
captionpos=b,
%
% flexible columns
basewidth={2em, 0.5em},
columns=flexible,
%
% Style for (listings') identifiers
identifierstyle={\ttfamily\color{black}},
% Style for declaration keywords
keywordstyle=[1]{\ttfamily\bfseries\color{dkviolet}},
% Style for gallina keywords
keywordstyle=[2]{\ttfamily\bfseries\color{dkgreen}},
% Style for sorts keywords
keywordstyle=[3]{\ttfamily\bfseries\color{ltblue}},
% Style for tactics keywords
keywordstyle=[4]{\ttfamily\color{dkblue}},
% Style for terminators keywords
keywordstyle=[5]{\ttfamily\color{dkred}},
%Style for iterators
%keywordstyle=[6]{\ttfamily\color{dkpink}},
% Style for strings
stringstyle=\ttfamily,
% Style for comments
commentstyle={\ttfamily\itshape\color{dkgreen}},
%
%moredelim=**[is][\ttfamily\color{red}]{/&}{&/},
literate=
    {fun}{{\color{dkgreen}{$\lambda\;$}}}1
    {bool}{{$\mathbb{B}$}}1
    {nat}{{$\mathbb{N}$}}1
    {Vforall2}{Vforall2}1 % quick workardoun to avoid partial replacement of 'forall' in identifier
    {nat\_equiv}{nat\_equiv}1 % quick workardoun to avoid partial replacement of 'nat' in identifier
    {forall}{{\color{dkgreen}{$\forall\;$}}}1
    {exists}{{$\exists\;$}}1
    {<-}{{$\leftarrow\;\;$}}1
    {=>}{{$\Rightarrow\;\;$}}1
    {==}{{\texttt{==}\;}}1
    {==>}{{$\Longrightarrow\;\;$}}1
%    {:>}{{\texttt{:>}\;}}1
    {->}{{$\rightarrow\;\;$}}1
    {<-->}{{$\longleftrightarrow\;\;$}}1
    {<->}{{$\leftrightarrow\;\;$}}1
    {<==}{{$\leq\;\;$}}1
    {\#}{{$^\star$}}1 
    {\\o}{{$\circ\;$}}1 
%    {\@}{{$\cdot$}}1 
    {\/\\}{{$\wedge\;$}}1
    {\\\/}{{$\vee\;$}}1
    {++}{{\texttt{++}}}1
    {~}{{\ }}1
    {¬}{{$\lnot$}}1     % this does not work
    {\@\@}{{$@$}}1
    {\\mapsto}{{$\mapsto\;$}}1
    {\\hline}{{\rule{\linewidth}{0.5pt}}}1
%
}[keywords,comments,strings]

\lstnewenvironment{coq}{\lstset{language=Coq}}{}

% pour inliner dans le texte
\def\coqe{\lstinline[language=Coq, basicstyle=\small]}
% pour inliner dans les tableaux / displaymath...
\def\coqes{\lstinline[language=Coq, basicstyle=\scriptsize]}

%%% Local Variables: 
%%% mode: latex
%%% Local IspellDict: british
%%% TeX-master: "main.tex"
%%% End: 

\title{Formal Verification of Computer Programs}
\subtitle{A Primier}
% \date{\today}
\date{\date{}}
\author[shortname]{Vadim Zaliva \textsuperscript{1} \and Nika Pona \inst{2}}
\institute[shortinst]{\textsuperscript{1}Carnegie Mellon  University \and \inst{2} Digamma.ai}

% \titlegraphic{\hfill\includegraphics[height=1.5cm]{logo.pdf}}

\begin{document}

\maketitle

\begin{frame}{Outline}
    \begin{enumerate}
        \item What are formal methods?
        \item Proof assistants (Coq) mini-intro
        \item Reasoning about programs (semantics, specification, etc)
        \item Motivating example (strtoimax\_lim)
        \item Detailed example (strlen of similar)
        \item Other languages? (Javascript, LLVM, etc.)
        \item Other approaches (exraction)
        \item Conclusions
    \end{enumerate}
\end{frame}

\begin{frame}{Table of contents}
  \setbeamertemplate{section in toc}[sections numbered]
  \tableofcontents[hideallsubsections]
\end{frame}



\section{What are formal methods?}
\begin{frame}{Formal verification - quick intro}
 
  We want to have high assurance that our code works as intended. One of the methods is formal verification. This means we want to produce a formal proof that our code works as intended. What does it mean exactly and how do we do it?
  
  \begin{enumerate}
  \item We start by writing a \emph{specification} in a formal language (like Coq or HOL) which strictly defines how our program should behave.
  \item Then we define the \emph{semantics} of our program which describes how it actually behaves. 
  \item Finally we mathematically prove that semantics of our program matches our specification.
\end{enumerate}
  
\end{frame}

\begin{frame}{Coq intro}
  As a formal language we choose {\bf Gallina}, mechanized version of Calculus of Inductive Constructions (aka dependent type theory), which is a very expressive theory well studied in mathematical logic.


  \smallskip
  
  It is much more likely to make a mistake in a formal proof (which is typically way longer than the code), so we want assurance that our proof is correct.
  \smallskip
  
  Hence we use a proof assistant Coq: a program that checks that your proof is correct. It also provides an environment to make the construction of proofs easier. 
  
  
\end{frame}


\section{Motivating Example}
\begin{frame}{\texttt{asn\_strtoimax\_lim}}
In this talk we are concentrating on formal verification of existing imperative programs. We took a function \texttt{asn\_strtoimax\_lim} from \texttt{asn1c} compiler to test our approach on a real-life function. Informal specification in the comments: 

 \begin{quote}
 Parse the number in the given string until the given *end position,
 returning the position after the last parsed character back using the
 same (*end) pointer.
 WARNING: This behavior is different from the standard strtol/strtoimax(3).
\end{quote}

The code:
 \url{https://github.com/vlm/asn1c/blob/9f470d60faf6b994de6b9d6c0dbfb9279d9bb48d/skeletons/INTEGER.c#L1033}
\end{frame}

\begin{frame}{\texttt{asn\_strtoimax\_lim}}


While doing the correctness proof we found a bug, can you see it?

  \url{https://github.com/vlm/asn1c/issues/344}
  
\end{frame}



\begin{frame}{\texttt{asn\_strtoimax\_lim}}
  Fixed version: \url{https://github.com/vlm/asn1c/blob/b361194599b46d97a398195e6a18f1d581f7fab9/skeletons/INTEGER.c#L1033}

  Is this fix OK?
  
\end{frame}

\section{Detailed example}


\begin{frame}[fragile]{Factorial example}
  Mathematical specification of factorial is recursive equation, for ($0 \leq n$) :
  \[ fact(0) = 1 \]
  \[ fact(n + 1) = fact(n)*(n+1) \]
  
  We can write it in Coq as a fixpoint (recursive) definition:
  \begin{lstlisting}[language=Coq]

  Fixpoint fact (n : nat) : nat :=
    match n with
    | O => 1
    | S n' => (S n') * fact n'
    end.

  \end{lstlisting}
  
  Note that this definition is also a functional program.
 
\end{frame}

\begin{frame}{Coq quick-intro}

Showcase some basic Coq stuff. (Fixpoint, Inductive defs: nat, Prop )
    
\end{frame}

\begin{frame}[fragile]{Factorial example: veryfing a functional program}

  We can write a more efficient functional program to compute factorial

  \begin{lstlisting}[language=Coq]

  Fixpoint fact_acc (n : nat) (acc : nat) :=
    match n with
    | 0 => acc
    | S k => fact_acc k (n * acc)
    end.

  Definition fact' (n : nat) :=
    fact_acc n 1.

  \end{lstlisting}

  Now we want to show that it actually computes factorial. To do this we can show in Coq that:
   \begin{lstlisting}[language=Coq]
  Theorem fact'_correct : forall n, fact' n = fact n.
  \end{lstlisting}
 
\end{frame}

\begin{frame}{Factorial example: veryfing a functional program}
Now using Coq's extraction mechanism we can automatically extract an OCaml or Haskell function that is provably correct. Alternatively, one could easily embed a functional language into Coq and reason about the existing implementation. What if you want to verify code written in imperative language?
  \end{frame}


  \begin{frame}{C light syntax and semantics: CompCert}

    First, you need to embed the language into Coq, meaning model its syntax and semantics in Coq. Luckily, this has been already done in the project called CompCert, a verfied compiler for C, almost entirely written in Coq and proved to work according to the specification (\url{http://compcert.inria.fr/}).

    \begin{quote}
      The striking thing about our CompCert results is that the middle-end bugs we found in all other compilers are absent. As of early 2011,the under-development version of CompCert is the only compiler we have tested for which Csmith cannot find wrong-code errors.
    \end{quote} ({\it Finding and Understanding Bugs in C Compilers}, Yang et al., 2011)

  \end{frame}
    
  \begin{frame}{Verifying Imperative program}
    So using CompCert our approach is as follows:
  \begin{itemize}
  \item parse C code into an abstract syntax tree using C light generator of CompCert
  \item write a functional specification using CompCert's model of C light 
  \item reason about the C light program using operational semantics defined in CompCert
\end{itemize}
\end{frame}


\begin{frame}[fragile]{Verifying Imperative Programs}

 Factorial C implementation that we want to verify

\begin{lstlisting}

unsigned int factorial (unsigned int input) {
  unsigned int output = 1;
  while (input){
      output = output*input ;
      input = input - 1 ;
    } 
  return output ;
}

\end{lstlisting}

The specification stays the same. 

\end{frame}

\begin{frame}[fragile]{Mechanized Syntax and Semantics of C programs}
C light AST corresponding to the factorial function: 
  \begin{lstlisting}[language=Coq]
(Ssequence
  (* int output = 1 *)
  (Sset _output (Econst_int (Int.repr 1) tuint)) 
  (Ssequence 
    (Swhile 
      (Etempvar _input tuint) (* while (input) *)
      (Ssequence  
        (Sset _output
          (Ebinop Omul (Etempvar _output tuint) 
          (* output = output*input *)
          (Etempvar _input tuint) tuint)) 
        (Sset _input
          (Ebinop Osub (Etempvar _input tuint) 
          (* input = input - 1 *)
          (Econst_int (Int.repr 1) tuint) tuint))))
            (* return output *)
    (Sreturn (Some (Etempvar _output tuint))))) |}.
     \end{lstlisting}
     
\end{frame}


\subsection{Operational semantics: bigstep}

  \begin{frame}{Operational Semantics}
    Our goal is to prove that programs written in C light behave as intented. To do this we need to formalize the notion of meaning of a C light program. We do this using what is called {\bf operational semantics}.

    \bigskip
    
    An operational semantics is a mathematical model of programming language execution. It is basically an interpreter defined formally.
    \bigskip
    
    We use big-step operational semantics used for all intermediate languages of CompCert.
  \end{frame}
  \begin{frame}{Operational Semantics}
    We assign primitive values to constants and then compositionally compute values of expressions and outcomes of statements.
   \begin{itemize}
    \item Each syntactic element (expressions and statements) is related to the indended result of executing this element.    
    \item Expressions are deterministically mapped to memory locations or values (integers, bool etc).
    \item The execution of statements depends on memory state and values stored in the local environment and produces {\bf outcomes} (break, normal, return), updated memory and local environment. Moreover, {\bf trace} of external calls is recorded.
      \end{itemize}
    \end{frame}
    \begin{frame}
      Go to factorial tutorial: \url{~/asn1verification/doc/tutorial/MiscExamples/factorial}
    \end{frame}

    \begin{frame}
      Going back to our first example:
      
       We wrote a formal specification of the function (based on the comment and analysis of the function), produced C light AST of the function using C light generator of CompCert and proved that the resulting AST evaluates to correct values on all valid inputs using operational semantics. 
      \end{frame}

\section{Other languages}

\begin{frame}{Other languages}

  \url{http://www.jscert.org/} JSCert: Certified JavaScript
  
  \end{frame}

\begin{frame}[allowframebreaks]{References}

  \bibliography{demo}
  \bibliographystyle{abbrv}

\end{frame}

\end{document}
