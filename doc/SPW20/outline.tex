
\documentclass{article}

\usepackage[usenames,dvipsnames]{xcolor}


\usepackage{listings}
\usepackage{multicol}


\lstset{
    frame=tb, % draw a frame at the top and bottom of the code block
    tabsize=4, % tab space width
    showstringspaces=false, % don't mark spaces in strings
    numbers=left, % display line numbers on the left
    commentstyle=\color{green}, % comment color
    keywordstyle=\color{blue}, % keyword color
    stringstyle=\color{red}, % string color
    identifierstyle=\color{grey},
}


\definecolor{diffstart}{named}{gray}
\definecolor{diffincl}{named}{OliveGreen}
\definecolor{diffrem}{named}{red}

\lstdefinelanguage{diff}{
    morecomment=[f][\color{diffstart}]{@@},
    morecomment=[f][\color{diffincl}]{+},
    morecomment=[f][\color{diffrem}]{-},
  }

%% https://github.com/nickgian/thesis/blob/master/lstcoq.sty
\usepackage{color}

\definecolor{ltblue}{rgb}{0,0.4,0.4}
\definecolor{dkblue}{rgb}{0,0.1,0.6}
\definecolor{dkgreen}{rgb}{0,0.35,0}
\definecolor{dkviolet}{rgb}{0.3,0,0.5}
\definecolor{dkred}{rgb}{0.5,0,0}

% lstlisting coq style (inspired from a file of Assia Mahboubi)
%
\lstdefinelanguage{Coq}{ 
%
% Anything betweeen $ becomes LaTeX math mode
mathescape=true,
%
% Comments may or not include Latex commands
texcl=false, 
%
% Vernacular commands
morekeywords=[1]{Section, Module, End, Require, Import, Export,
  Variable, Variables, Parameter, Parameters, Axiom, Hypothesis,
  Hypotheses, Notation, Local, Tactic, Reserved, Scope, Open, Close,
  Bind, Delimit, Definition, Let, Ltac, Fixpoint, CoFixpoint, Add,
  Morphism, Relation, Implicit, Arguments, Unset, Contextual,
  Strict, Prenex, Implicits, Inductive, CoInductive, Record,
  Structure, Canonical, Coercion, Context, Class, Global, Instance,
  Program, Infix, Theorem, Lemma, Corollary, Proposition, Fact,
  Remark, Example, Proof, Goal, Save, Qed, Defined, Hint, Resolve,
  Rewrite, View, Search, Show, Print, Printing, All, Eval, Check,
  Projections, inside, outside, Def},
%
% Gallina
morekeywords=[2]{forall, exists, exists2, fun, fix, cofix, struct,
  match, with, end, as, in, return, let, if, is, then, else, for, of,
  nosimpl, when},
%
% Sorts
morekeywords=[3]{Type, Prop, Set, true, false, option},
%
% Various tactics, some are std Coq subsumed by ssr, for the manual purpose
morekeywords=[4]{pose, set, move, case, elim, apply, clear, hnf,
  intro, intros, generalize, rename, pattern, after, destruct,
  induction, using, refine, inversion, injection, rewrite, setoid_rewrite, congr,
  unlock, compute, ring, field, fourier, replace, setoid_replace, fold, unfold,
  change, cutrewrite, simpl, have, suff, wlog, suffices, without,
  loss, nat_norm, assert, cut, trivial, revert, bool_congr, nat_congr,
  symmetry, transitivity, auto, split, left, right, autorewrite},
%
% Terminators
morekeywords=[5]{by, done, exact, reflexivity, tauto, romega, omega,
  assumption, solve, contradiction, discriminate},
%
% Control
morekeywords=[6]{do, last, first, try, idtac, repeat},
%
% Comments delimiters, we do turn this off for the manual
morecomment=[s]{(*}{*)},
%
% Spaces are not displayed as a special character
showstringspaces=false,
%
% String delimiters
morestring=[b]",
morestring=[d]’,
%
% Size of tabulations
tabsize=3,
%
% Enables ASCII chars 128 to 255
extendedchars=false,
%
% Case sensitivity
sensitive=true,
%
% Automatic breaking of long lines
breaklines=false,
%
% Default style fors listings
basicstyle=\small,
%
% Position of captions is bottom
captionpos=b,
%
% flexible columns
basewidth={2em, 0.5em},
columns=flexible,
%
% Style for (listings') identifiers
identifierstyle={\ttfamily\color{black}},
% Style for declaration keywords
keywordstyle=[1]{\ttfamily\bfseries\color{dkviolet}},
% Style for gallina keywords
keywordstyle=[2]{\ttfamily\bfseries\color{dkgreen}},
% Style for sorts keywords
keywordstyle=[3]{\ttfamily\bfseries\color{ltblue}},
% Style for tactics keywords
keywordstyle=[4]{\ttfamily\color{dkblue}},
% Style for terminators keywords
keywordstyle=[5]{\ttfamily\color{dkred}},
%Style for iterators
%keywordstyle=[6]{\ttfamily\color{dkpink}},
% Style for strings
stringstyle=\ttfamily,
% Style for comments
commentstyle={\ttfamily\itshape\color{dkgreen}},
%
%moredelim=**[is][\ttfamily\color{red}]{/&}{&/},
literate=
    {fun}{{\color{dkgreen}{$\lambda\;$}}}1
    {bool}{{$\mathbb{B}$}}1
    {nat}{{$\mathbb{N}$}}1
    {Vforall2}{Vforall2}1 % quick workardoun to avoid partial replacement of 'forall' in identifier
    {nat\_equiv}{nat\_equiv}1 % quick workardoun to avoid partial replacement of 'nat' in identifier
    {forall}{{\color{dkgreen}{$\forall\;$}}}1
    {exists}{{$\exists\;$}}1
    {<-}{{$\leftarrow\;\;$}}1
    {=>}{{$\Rightarrow\;\;$}}1
    {==}{{\texttt{==}\;}}1
    {==>}{{$\Longrightarrow\;\;$}}1
%    {:>}{{\texttt{:>}\;}}1
    {->}{{$\rightarrow\;\;$}}1
    {<-->}{{$\longleftrightarrow\;\;$}}1
    {<->}{{$\leftrightarrow\;\;$}}1
    {<==}{{$\leq\;\;$}}1
    {\#}{{$^\star$}}1 
    {\\o}{{$\circ\;$}}1 
%    {\@}{{$\cdot$}}1 
    {\/\\}{{$\wedge\;$}}1
    {\\\/}{{$\vee\;$}}1
    {++}{{\texttt{++}}}1
    {~}{{\ }}1
    {¬}{{$\lnot$}}1     % this does not work
    {\@\@}{{$@$}}1
    {\\mapsto}{{$\mapsto\;$}}1
    {\\hline}{{\rule{\linewidth}{0.5pt}}}1
%
}[keywords,comments,strings]

\lstnewenvironment{coq}{\lstset{language=Coq}}{}

% pour inliner dans le texte
\def\coqe{\lstinline[language=Coq, basicstyle=\small]}
% pour inliner dans les tableaux / displaymath...
\def\coqes{\lstinline[language=Coq, basicstyle=\scriptsize]}

%%% Local Variables: 
%%% mode: latex
%%% Local IspellDict: british
%%% TeX-master: "main.tex"
%%% End: 

\begin{document}

\title{Verifying existing ASN.1 compiler}

\author{Nika Pona, Vadim Zaliva \\
        Digamma.ai}

\maketitle

\begin{abstract}

\end{abstract}

\begin{enumerate}

\item Intro
  \begin{itemize}
  \item ASN.1 intro and example 
  \item Mention some common problems with ASN.1 parsers
  \end{itemize}
\item Approaches to verification of parsers
  \begin{itemize}
  \item Functional parsers (Narcissus)
  \item Extraction to C, partial correctness (Everparse, Galois)
  \end{itemize}

  \item Example of strtoimax of asn1c
\begin{itemize}
  \item Operational semantics proof (CompCert, C semantics)
  \item VST proof 
    \item Bugs discovered
  \item Lessons learned
  \end{itemize}

\item Current work


\end{enumerate}

\section{Veryfing a function from \texttt{asn1c} compiler}

  At \url{Digamma.ai} we are working on formal verification of existing {\bf imperative} programs using Coq. We took a function \texttt{asn\_strtoimax\_lim} from \texttt{asn1c} compiler to test several approaches on a simple real-life example. The function is relatively simple, but at the same time uses many features of C that make verifying imperative programs challenging. XER decoding functions for INTEGER, OBJECT-IDENTIFIER and RELATIVE-OID types critically depend on this function. XER decoding is pretty straightforward\footnote{INTEGERs are encoded in text as their decimal representation, OBJECT IDENTIFIERs are encoded as decimal numbers in the order of the object identifier's components.}, however, we found that it is compromised by three bugs in this function.
    Informal specification from the comments: 
  \begin{quote}

 { \it Parse the number in the given string until the given *end position,
 returning the position after the last parsed character back using the
 same (*end) pointer.
 WARNING: This behavior is different from the standard strtol/strtoimax(3). }
\end{quote}
  
  {\fontsize{8}{4}\selectfont  \lstinputlisting[language=C]{asn_strtoimax_lim_old.c}}
  \skip
  
\paragraph{Negative range bug}

When we go beyond allowed \textit{int} range, a wrong result is given for some inputs\footnote{Assume we are working on a 8-bit system and maximal signed int \texttt{MAX\_INT} is 127}:

\skip
\begin{tabular}{|l|l|}
 \hline
input & \texttt{-128} \\
intmax & \texttt{\ 127}\\
upper boundary& \texttt{\ 12} \\
last digit max& \texttt{\ 7}\\
return & \color{green}\texttt{\ ASN\_STRTOX\_ERROR\_RANGE}\\
\hline 
input & \texttt{-1281} \\
intmax & \texttt{\ 127}\\
upper boundary& \texttt{\ 12} \\
last digit max& \texttt{\ 7}\\
return & \color{red}\texttt{\ -127, ASN\_STRTOX\_OK}\\
\hline
    \end{tabular}

 \skip
 
This happens whenever the input string represents a number smaller than \texttt{MIN\_INT}, due to the fact that absolute value of \texttt{MIN\_INT} is greater than \texttt{MAX\_INT}, thus negative number cannot be treated as $\mathrm{value}\times\mathrm{sign}$ when $\mathrm{value}$ is represented as \textit{int}. The bug  was filed and promptly fixed by developers:

   
  
  {\fontsize{8}{4}\selectfont \lstinputlisting[language=diff]{bug1fix.diff}}

  
  \paragraph{Memory store bug}
  We uncovered a curious bug in the \emph{fixed} version. Note that we store \texttt{str} before reading from \texttt{str}: \newpage
  
     \begin{lstlisting}[language=C]
  if(str < *end) {
    *end = str;
    if(*str >= 0x30 && *str <= 0x39){
      return ASN_STRTOX_ERROR_RANGE;
    } else {
      *intp = sign * value;
      return ASN_STRTOX_EXTRA_DATA;
    }
  }
  \end{lstlisting}
  
     We can construct an example where we get wrong results by overwriting data in \texttt{str}. Let minimal signed int \texttt{MIN\_INT} = $-4775808$ and \texttt{*str} = {\color{red}2d 34 37 37 35 38 30 31 31 31} (stands for ``-477580111''). 

\begin{multicols}{2}
  
    {\bf Scenario 1}:

    Assume that \texttt{*end = str + 9} and

    \texttt{end} $\geq$ $\texttt{str + 9}$.

    {\color{red}2d 34 37 37 35 38 30 {\color{blue}$\overbrace{31}^{\texttt{str + 7}}$} 31 31

      $\ldots$ $\overbrace{\texttt{X}}^{\texttt{end}}$}

    Then at \texttt{str + 7} we store

    \texttt{*end = (str + 7)}

     Let \texttt{str + 7 = 21 21 21 26} 

     {\color{red}2d 34 37 37 35 38 30 31 {\color{blue} $\overbrace{31}^{\texttt{str + 8}}$} 31

       $\ldots$ $\overbrace{\texttt{21 21 21 26}}^{\texttt{end}}$}
     
     
     And since at \texttt{str + 8} we read `1'
     

     
     The output is  {\color{green}\texttt{ASN\_ERROR\_RANGE}}
     \columnbreak
     
    {\bf Scenario 2}:

    Assume that \texttt{*end = str + 9} and

    \texttt{end = str + 7}:
    
    {\color{red}2d 34 37 37 35 38 {\color{blue} $\overbrace{30 \; 31 \; 31 \; 31}^{\texttt{end}}$}}
    

    Then at \texttt{str + 7} we store

    \texttt{*end = str + 7}

     Let \texttt{str + 7 = 21 21 21 26} 

   
    {\color{red}2d 34 37 37 35 38 30 {\color{blue}$\overbrace{21 \; 21 \; 21 \; 26}^{\texttt{end}}$}}

    (stands for $``-477580!!!\&''$)

    And since at \texttt{str + 8} we read `!' 

    The output is {\color{red}\texttt{ASN\_EXTRA\_DATA}}
    \end{multicols}
    

Hence, when the value of the \texttt{end} pointer is treated as a part of the input data, there is a bug where the resulting error value could be incorrect. On the other hand, it is hard to think of a legitimate use-case where the pointer would be a part the input data. Under such interpretation, there is an implicit pre-condition in the specification, mandating that:

{\small
\texttt{(*end < end) || (end + sizeof(const char *) <= str)}}


\paragraph{Specification ``bug''}
After addressing the two bugs we discovered we were able to successfully verify that the function finally corresponds to the specification we wrote for. However, it was noticed the following behavior:

For input \texttt{``a''} it stores value 0 and returns {\color{green}\texttt{ASN\_STRTOX\_EXTRA\_DATA}} (same behaviour as on input \texttt{``0a''}), which could be unintended by authors. \\

This code was part of the library for 15 years. The library is covered by extensive unit and randomized tests. It is used in production by multiple users. Yet, the vulnerabilities are there and pose potential problems.

\section{Our approach}

\paragraph{Machine arithmetic} The first bug is related to data type ranges and modulo integer arithmetic. These sort of problems are fairly common and require careful coding to be avoided. Formal verification enforces a strict mathematical model of all computer arithmetic and invariably exposes all such bugs. We use CompCert's \texttt{Integer} library that provides theory of 8-, 16-, 32- and 64-bit integers and 32- and 64-bit pointers. It has some basic lemmas, however, doing proofs by hand is quite tedious, however automation can be easily achieved here for programs that allow no overflow, since then one can use tactics for solving comparisons on Z (if overflow is allowed, it is an NP-complete problem and human input may be needed). VST provides tactics for automating such proofs. Here we can hope to achieve almost complete automation.

\paragraph{Memory safety} The second problem was related to \textit{pointer aliasing}. These problems are not immediately obvious because C language does not allow us to enforce any memory aliasing restrictions (unlike, say Rust). In formal verification, there is a rigorous model to analyze such kind of problems called \textit{separation logic}. VST provides some automation related to this, however, it is not sufficient, but one can build custom libraries and if restricted to specific domain separation logic proofs can be fully automated.

    \paragraph{Specification} The third issue shows us that your formal verification is only as good as your specification. In this domain we not only have a typical problem of how to validate specification, but also: what kind of specifications make proofs simpler. VST allows to completely separate functional spec from memory spec. However, we noticed that using a C-like spec significantly reduces proof effort and pushes it to the functional level. 
\end{enumerate}

\subsection{Operational semantics proof}

\subsection{VST proof} 

\section{Intro}
  ASN.1 intro and simple example 

  Some common problems with ASN.1 parsers

The Computer Vulnerabilities and Exposures (CVE) database [7a] lists critical ASN.1-related bugs found each year in the existing systems, and there have already been noteworthy exposures [8] that although not as dire to security as first feared [9] clearly spell out a first awareness of the vast risk and exposure. We analyzed the last 4 years of ASN.1-related issues reported in Computer Vulnerabilities and Exposures (CVE) database [7b]. Among vulnerabilities studied were CVEs for various software and hardware products and vendors, including Apple, axTLS, Botan, Bounty Castle, librcrypto++, libtasn, LibTomCrypt, Linux Kernel, MatrixSSL, Mozilla NSS (Firefox), Objective Systems, OpenSSL, PolarSSL, RSA BSAFE, Samba, Samsung, Snapdragon, strongSwan, and Wireshark. It was found that 39 out of 52 problems analyzed were related to memory safety, 6 related to stack and heap bounds checking, and 3 related to issues caused by applications accepting not well-formed ASN.1 input. Having proved just six formal properties would have prevented 49 out of 52 vulnerabilities, that is, more than 90% of reported vulnerabilities.

\section{Approaches to verification of parsers}

  Functional parsers (Narcissus)

  Extraction to C (Everparse, Galois)

  Project Everest [19] is our most direct thematic competition, and some of the high-level Project Everest documentation makes passing mention to ASN.1. The Project Everest stewards adopted a different approach based on F* [4], and the Project Everest ASN.1 work  appears at best still in far distant plans.

It is also noted that Galois did some work on ASN.1 verification in the past (circa 2012) [11]. It appears that Galois abandoned [12] the goal of full ASN.1 verification that we pursue in our project with our more pragmatic approach; Galois is now only exploring a limited subset ASN.1 verification adequate for the “vehicle-to-vehicle” (V2V) market [13], but that particular subset has limited broader applicability and the Galois effort appears encumbered by aspects of an unsuccessful  approach. It is noted that although our goal is eventually to verify all ASN.1, we decided to start with a different (X.509-related) subset of ASN.1; this subset is reasonably small, but X.509 is so widely used that our initial verified implementation of our chosen subset will have a large volume and wide range of immediate commercial applications.

\section{Our approach}
  Our technology and approach will provide an entirely new level of software and protocol verification employing recently-emergent provable full-functional correctness methods.
Systems and methods to date either (a) automatically tested but not formally verified (b) use verification approach which rely on automatic extraction from executable specifications (for example involving network stack synthesis [31], optimizing compilers [3], cryptographic libraries [6], and encoder/decoders [32]), or (c) apply a form of formal verification which only proves partial correctness properties (partial verification of NAT stack only proving parts of DPDK are specification compliant [33], partial verification of Linux kernel TCP implementation with 55\% line coverage and 92\% protocol coverage [34]). Consequently, (a) and (c) do not provide sufficient correctness guarantees, while (c) is often impractical due to poor performance and compatibility limitations. In contrast, we pursue a far deeper and comprehensive verification approach to performance and portability and seek to prove actual industrial-level C-code implementation.
  
\subsection{Future work}
  Formalize the X.509 part of the ASN.1 standard.  
   
  Refactor ASN1C code to make it suitable for verification.

  Establish correctness of encoders/decoders for primitive types.

  Establish correctness of encoders/decoders for constructed types.

  Prove high-level properties.

  Experiment with OCaml code extraction from Executable Specifications.

  Establish metrics and evaluate code bases to estimate the effort required to prove the remainder of ASN.1 stack.

  Produce the final ASN1C code and associated documentation in the form of commercial product that could be sold.






\end{document}
